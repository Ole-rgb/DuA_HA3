\documentclass[a4paper,12pt]{scrartcl}
%%%%%%%%%%%%%%%%%%%%%%%%%%%%%%%%%%%%%%%%%
% Bitte zuerst diese Variablen setzen!
%
\def\Nr{3}					%Hier die Nummer des Übungsblatts eintragen
\def\NameA{Alice Müller}	%Hier den Name eintragen
\def\MatA{1234567}			%Hier die Matrikelnummer eintragen
\def\NameB{Bob Meier}		%Hier den Name des 2. Gruppenmitglieds eintragen
\def\MatB{765421}			%Hier die Matrikelnummer des 2. Gruppenmitglieds eintragen
\def\NameC{}				%Name 3. Gruppenmitglied
\def\MatC{}					%Matrikelnummer 3. Gruppenmitglied
\def\NameD{}				%Name 4. Gruppenmitglied
\def\MatD{}					%Matrikelnummer 4. Gruppenmitglied
%
%%%%%%%%%%%%%%%%%%%%%%%%%%%%%%%%%%%%%%%%%
\usepackage[utf8]{inputenc}
\usepackage[ngerman]{babel}
\usepackage{hyperref}
\usepackage{booktabs}
\usepackage{geometry}
\usepackage{amssymb}
\usepackage{amsmath}
\usepackage{ifthen}
\usepackage{enumerate}
\usepackage{verbatim}
\usepackage{multicol}
\usepackage{algorithm}
\usepackage{xcolor}
\setlength{\marginparwidth}{2.5cm}
\usepackage{todonotes}
\usepackage{tikz}
\usepackage{forest}
\usetikzlibrary{
	arrows,
	arrows.meta,
	automata,
	calc,
	chains,
	trees,
	positioning,
	scopes,
	decorations.pathmorphing,
	shapes,
	backgrounds,
	chains,
	}
\tikzset{%
    every node/.style={minimum size = 6mm, circle, draw, inner sep=0cm, text centered},
    no edge/.style={edge from parent/.append style={draw=none}},
}

\geometry{margin=3cm, top=2.7cm}
\renewcommand{\thesection}{\arabic{section}{.}}

\begin{document}
\begin{center}
	\sffamily
	\bfseries
	\LARGE
	Datenstrukturen und Algorithmen\\
	\Large
	\vspace{.2\parskip}
	Hausübung \Nr\\
	\normalsize\normalfont
	WiSe 23/24
	\vspace{.2\parskip}
\end{center}

\begin{tabular}[t]{p{4.5cm} p{3cm}}
	\toprule
	Name & Matrikelnummer\\
	\midrule
	\NameA & \MatA\\
	\NameB & \MatB\\
	\ifx\NameC\empty\else\NameC & \MatC\\\fi
	\ifx\NameD\empty\else\NameD & \MatD\\\fi
	\bottomrule
\end{tabular}
\hfill
\begin{tabular}[t]{ccccc}
	\toprule
	A1 & A2 & A3 & Bonus & $\Sigma$ \\
	\midrule
	\\
	\bottomrule	
\end{tabular}
\hfill\\


%%%%%%%%%%%%%%%%%%%%%%%%%%%%%%%%%%%%%%%%%%%%%%%%%%%%%%%%%%%%%%%%%%%%%%%%%%%%%%%%%%%%%%%%%
% Ab hier bearbeiten 
%%%%%%%%%%%%%%%%%%%%%%%%%%%%%%%%%%%%%%%%%%%%%%%%%%%%%%%%%%%%%%%%%%%%%%%%%%%%%%%%%%%%%%%%%
Informationen zu \LaTeX{} auf z.B.: \url{https://tex.cloud.uni-hannover.de/learn}.

\section*{Aufgabe 1}
\begin{enumerate}[a)]
	\item
    \item \
    \begin{center}
        \begin{tikzpicture}[
            level distance = 0.9cm,
            level/.style={sibling distance=8cm/2^#1},
            ]
            \node {15}
                child{ node {11}
                    child{ node {4} 
                        child{ node {1}}
                        child{ node {6}}
                    }
                    child{ node {12}}
                }
                child{ node {29} 
                    child{ node {27}
                        child{ node {21}}
                        child{[no edge] node[draw=none] {}}
                    }
                    child{ node {30}}                    
                };
        \end{tikzpicture}
        \end{center}
\end{enumerate}


\section*{Aufgabe 2}
\begin{enumerate}[a)]
	\item
    $$   
        \def\h#1{\parbox{1.5em}{\centering #1}}
        \renewcommand{\arraystretch}{2}
        \begin{array}{|c|*{6}{c|}}\hline
            \h0 & \h1 & \h2 & \h3 & \h4 & \h5 & \h6\\ \hline
            & & & & & & \\ \hline
            & & & & & & \\ \hline
            & & & & & & \\ \hline
            & & & & & & \\ \hline
            & & & & & & \\ \hline
            & & & & & & \\ \hline
            & & & & & & \\ \hline
            & & & & & & \\ \hline
            & & & & & & \\ \hline
            & & & & & & \\ \hline
        \end{array}
        \quad
        \begin{array}{|c|*{6}{c|}}\hline
            \h0 & \h1 & \h2 & \h3 & \h4 & \h5 & \h6\\ \hline
            & & & & & & \\ \hline
            & & & & & & \\ \hline
            & & & & & & \\ \hline
            & & & & & & \\ \hline
            & & & & & & \\ \hline
            & & & & & & \\ \hline
            & & & & & & \\ \hline
            & & & & & & \\ \hline
            & & & & & & \\ \hline
            & & & & & & \\ \hline
        \end{array}
    $$
	\item
	\item
\end{enumerate}


\section*{Aufgabe 3}
\begin{enumerate}[a)]
	\item 
	\item
	\item Da 'dicegame()' die Funktion 'calculateprobability()' für Werte von 'sides + 1' bis '2 * sides + 1 ' aufruft, fokussieren wir uns auf die Laufzeit von 'calculateprobability()'.
    In der Funktion 'calculateprobability()' gibt es eine for-loop (von 1 bis x).
    Durch die Memorisierung wird jeder Wert von x nur einmal berechnet und sonst aus dem Dict ausgelesen.
    Dadurch beläuft sich die Laufzeit von 'calculateprobability()' auf O(n*x) mit x als aktueller Kandidat dessen Wahrscheinlichkeit berechnet werden soll und n der Anzahl der Seiten des Würfels.
    Da wir also alle Wahrscheinlichkeiten bis zu '2 * sides + 1' berechnen müssen ergibt sich eine Laufzeit von O((2*sides + 1)*sides).



 

\end{enumerate}
\end{document}